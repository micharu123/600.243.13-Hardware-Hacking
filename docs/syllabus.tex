\documentclass[12pt]{article}
\textwidth=7in
\textheight=9.5in
\topmargin=-1in
\headheight=0in
\headsep=.5in
\hoffset=-.85in
\pagestyle{empty}
\renewcommand{\thefootnote}{\fnsymbol{footnote}}

\begin{document}
\begin{center}
{\bf Introduction to Hardware Hacking\\}
{EN.600.243.13 \ \ MWTH 2:30 - 5:45 PM \ \  Shaffer 100}
\end{center}

\setlength{\unitlength}{1in}

\begin{picture}(6,.1) 
\put(-.25,0) {\line(1,0){7}}         
\end{picture}

 

\renewcommand{\arraystretch}{2}

\vskip.25in
\noindent\textbf{Instructors:} Michael Rushanan,  Malone Hall 328, micharu1/at/cs.jhu.edu, michaelrushanan.org 

\-\hspace{2cm}Paul Martin,  Malone Hall 328, pmartin/at/pauldmartin.org, pauldmartin.org

\vskip.25in
\noindent\textbf{Office Hours:} MW 2:00-2:30 PM and by appointment

\vskip.25in
\noindent\textbf{Textbook:}  No textbook required. 
\vskip.25in
\noindent\textbf{Prerequisites:} \footnotemark Intermediate Programming, Data Structures, CSF

\footnotetext{You may not necessarily need to meet these requirments.}

\vskip.25in
\noindent \textbf{Course Description}: Our favorite electronic devices, such as gaming consoles and smartphones, have a common root --- hardware. These deceptively simple interconnections of electronic components perform arithmetic and logic operations that enable our devices to interact with us and extend current methods for software security, communication, and marketing. In this course, we first survey hacks (e.g., rooting) that furnish a practical understanding of the hardware/software relationship. We continue this relationship by examining the use of hardware emulation and security extensions. Lastly, we explore other niche hardware uses such as asset tracking, advertising, and authentication.

\vspace*{.15in}
\noindent \textbf{Course Outline:} 

\begin{center} \begin{minipage}{5in}
\begin{flushleft}
Introduction, Team Formations, and Software Tools\dotfill 1/5\\
Prototyping and Open Source Hardware\dotfill 1/6\\
Basic Soldering and Consumer Devices\dotfill 1/8\\
Video Game System Troubleshooting and Mods\dotfill 1/12\\
Hardware Emulation and Virtualization\dotfill 1/13\\
RFID, Smartcards, NFC\dotfill 1/15\\
Hardware Security Extensions\dotfill 1/19\\
Internet-of-Things and Security Concerns\dotfill 1/20\\
Hackathon\dotfill 1/23\\
\end{flushleft}
\end{minipage}
\end{center}

\vspace*{.15in}
\noindent\textbf{Grade Policy:} Your grade is derived solely from attendance and participation. You are required to attend all classes, read all assigned reading, and maintain a blog (see Homework). Please email the course instructors if you plan on being absent.


\vskip.25in
\noindent \textbf{Course Objectives}: In this course you should: gain an appreciation of hardware and its use, accumulate tools for working and interfacing with embedded systems, acquire prototyping and soldering skills required for hobbyist-level projects, and hack your own hardware!


%\vskip.25in
%\noindent\textbf{Grading}: In class participation will be used to assess your understanding of materials presented in class and in the assigned readings. Participation is defined as in class discussion, peer discussion, and group experiments. 

\vskip.25in
\noindent\textbf{Homework}:  Various articles and papers will be assigned prior to the start of class. Per the Grade Policy, you must be prepared to actively engage in both class and peer discussions regarding the assigned reading. In addition, you must maintain a blog that highlights your work, in and out of class, and briefly summarizes assigned reading. You may use Piazza to post semi-privately (see Piazza) or any free blogging service (e.g., Blogger). 

\vskip.25in
\noindent\textbf{Piazza}:  We will be conducting all class-related discussion on Piazza this term. The quicker you begin asking questions on Piazza (rather than via emails), the quicker you'll benefit from the collective knowledge of your classmates and instructors. We encourage you to ask questions when you're struggling to understand a concept—you can even do so anonymously.

\vskip.25in
\noindent\textbf{CS Department Academic Integrity}:  \textit{``The strength of the university depends on academic and personal integrity. In your studies, you must be honest and truthful. Ethical violations include cheating on exams, plagiarism, reuse of assignments, improper use of the Internet and electronic devices, unauthorized collaboration, alteration of graded assignments, forgery and falsification, lying, facilitating academic dishonesty, and unfair competition.}

\textit{Academic honesty is required in all work you submit to be graded. Except where the instructor specifies group work, you must solve all homework and programming assignments without the help of others. For example, you must not look at any other solutions (including program code) to your homework problems or similar problems. However, you may discuss assignment specifications with others to be sure you understand what is required by the assignment.}

\textit{If your instructor permits using fragments of source code from outside sources, such as your textbook or on-line resources, you must properly cite the source. Not citing it constitutes plagiarism. Similarly, your group projects must list everyone who participated.}

\textit{Falsifying program output or results is prohibited.}

\textit{Your instructor is free to override parts of this policy for particular assignments. To protect yourself: (1) Ask the instructor if you are not sure what is permissible. (2) Seek help from the instructor or TA, as you are always encouraged to do, rather than from other students. (3) Cite any questionable sources of help you may have received.}

\textit{Students who cheat will suffer a serious course grade penalty in addition to being reported to university officials. You must abide by JHU's Ethics Code: Report any violations you witness to the instructor. You may consult the associate dean of students and/or the chairman of the Ethics Board beforehand. For more information, see the Undergraduate Academic Ethics Board website and the Procedures for Handling Allegations of Misconduct by Full-time \& Part-time Graduate Students.''}

\end{document}